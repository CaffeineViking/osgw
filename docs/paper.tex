\documentclass[a4paper, twocolumn]{article}
\usepackage[utf8]{inputenc}
\usepackage[T1]{fontenc}
\usepackage[pdftex, hidelinks,
            pdftitle={Real-Time Ocean Simulation and Rendering Using Gerstner Waves},
            pdfauthor={Erik Sven Vasconcelos Jansson},
            pdfsubject={Computer Graphics -- Procedural Animation},
            pdfkeywords={real-time,ocean,simulation,rendering,
                         opengl,glsl,gerstner,wave}]{hyperref}

\usepackage{bm}
\usepackage{caption}
\usepackage{listings}
\usepackage{pdfpages}
\usepackage{booktabs}
\usepackage{mathtools}
\usepackage{blindtext}
\usepackage{algorithmic}
\usepackage{graphicx}
\usepackage{courier}
\usepackage{acronym}
\usepackage{amssymb}
\usepackage{amsthm}
\usepackage{siunitx}
\usepackage{algorithm}
\usepackage[capitalize, noabbrev]{cleveref}
\usepackage[activate={true, nocompatibility}, final,
            tracking=true, kerning=true, spacing=true,
            factor=1100, stretch=10, shrink=10]{microtype}

\DeclareCaptionFormat{modifiedlst}{\rule{\linewidth}{0.85pt}\\[-2.9pt]#1#2#3}
\captionsetup[lstlisting]{format =  modifiedlst,
labelfont=bf,singlelinecheck=off,labelsep=space}
\lstset{basicstyle=\footnotesize\ttfamily,
        breakatwhitespace = false,
        breaklines = true,
        keepspaces = true,
        language = C++,
        showspaces = false,
        showstringspaces = false,
        frame = tb,
        numbers = left,
        numbersep = 5pt,
        xleftmargin = 16pt,
        framexleftmargin = 16pt,
        belowskip = \bigskipamount,
        aboveskip = \bigskipamount,
        escapeinside={<@}{@>}}

\title{\vspace{-1.5em}\textbf{Real-Time Ocean Simulation and \\
                              Rendering Using Gerstner Waves}}
\author{{\textbf{Erik Sven Vasconcelos Jansson}} \\
        {\href{mailto:erija578@student.liu.se}
        {\texttt{<erija578@student.liu.se>}}} \\
        {Linköping University, Sweden}}

\begin{document}

    \maketitle

    \begin{abstract} \input{sections/abstract.tex} \end{abstract}
    \begin{figure}[h]
        \includegraphics[width=\linewidth]{figures/gerstner.png}
    \end{figure} \newpage
    \section{Introduction} \label{sec:introduction} \input{sections/introduction.tex}
    \section{Related Work} \label{sec:related_work} \begin{equation} \label{eq:sum_of_sine_waves}
    H(\mathbf{p},t) = \sum{
        A_i \sin \, (
            (\mathbf{p} \cdot \mathbf{d}_i) w_i
            + t \varphi_i
        )
    }
\end{equation}

\begin{equation} \label{eq:sum_of_sine_waves_position}
    \mathbf{H}(x,z,t) = \begin{bmatrix}
        x&
        H(\mathbf{p},t)&
        z
    \end{bmatrix}\;\,,\;\,
    \mathbf{p} = \begin{bmatrix}
        x&z
    \end{bmatrix}
\end{equation}

\begin{equation} \label{eq:sum_of_sine_waves_normal}
    \mathbf{\hat{H}}(x,z,t) = \begin{bmatrix}
        -\frac{\partial}{\partial x} H(\mathbf{p},t)&
        1&
        -\frac{\partial}{\partial z} H(\mathbf{p},t)
    \end{bmatrix}
\end{equation}

\begin{equation} \label{eq:gerstner_wave}
    G_j(\mathbf{p},t) = \sum{
        d_{i,j} Q_iA_i \cos \, (
            (\mathbf{p} \cdot \mathbf{d}_i) w_i
            + t \varphi_i
        )
    }
\end{equation}

\begin{equation} \label{eq:gerstner_wave_position}
    \mathbf{G}(x,z,t) = \begin{bmatrix}
        x + G_x(\mathbf{p},t)\\
        H(\mathbf{p},t)\\
        z + G_z(\mathbf{p},t)
    \end{bmatrix}\;\;\,,\;\;\,
    \mathbf{p} = \begin{bmatrix}
        x&z
    \end{bmatrix}
\end{equation}

\begin{equation} \label{eq:gerstner_wave_normal}
    \begin{split}
    \mathbf{\hat{G}}(x,z,t) &= \begin{bmatrix}
        -\sum{d_{i,x} A_iw_i \cos \psi_i}\\
        1 - \sum{Q_iA_iw_i \sin \psi_i}\\
        -\sum{d_{i,z} A_iw_i \cos \psi_i}
    \end{bmatrix}\;\;\;\,,\\
        \psi_i &= (\mathbf{G}(x,z,t) \cdot \mathbf{D}_i)w_i + t \varphi_i
    \end{split}
\end{equation}

    \section{Implementation} \label{sec:implementation} \input{sections/implementation.tex}
    \section{Results} \label{sec:results} \input{sections/results.tex}
    \section{Conclusions} \label{sec:conclusions} \input{sections/conclusions.tex}

    \newpage

    \section*{Acknowledgements}

    This project was developed as part of the course TNM084 -- Procedural Methods for Images in 2017 under the supervision of Stefan Gustavson at LiTH.

    I would like to thank Stefan Gustavson and Ian McEwan for providing the very useful Simplex Noise shader in \href{https://github.com/stegu/webgl-noise}{\texttt{webgl-noise}} under a permissive license. Also, Camilla Löwy for maintaining the handy \href{https://github.com/glfw/glfw}{\texttt{glfw}} library for portable window and context creation, and equally as much to Christophe Riccio for \href{https://github.com/g-truc/glm}{\texttt{glm}}!

    \section*{Source Code}

    All code is under the MIT license. You'll find it as \href{https://github.com/caffeineviking/osgw}{\texttt{osgw}} on GitHub, or by (yes!): \texttt{unzip osgw.pdf}.

    \nocite{*} % TODO: remove
    \bibliographystyle{abbrv}
    \bibliography{paper}

    \appendix
    \clearpage \onecolumn
    \lstinputlisting[caption={Gerstner Wave Shader for OpenGL with GLSL 4.10.},
                 morekeywords={vec2, vec3, uniform, uint, version, dot, sin, cos, inout},
                 label={lst:gerstner}]{listings/gerstner.glsl}


\end{document}
